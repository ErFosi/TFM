\chapter{Resultados}

Como resultado de este trabajo se ha realizado una demostración práctica para validar la arquitectura propuesta. Esta sección presenta el caso de uso implementado y los resultados de las pruebas ejecutadas.




\section{Resultados de Pruebas y Cobertura de Código}

Como parte de la validación del sistema, se han ejecutado las baterías de tests descritas en la sección de desarrollo. A continuación se presentan los resultados obtenidos.

\subsection{Validación Criptográfica}

Los resultados de la validación criptográfica confirman la correctitud de las implementaciones:

\begin{itemize}
    \item \textbf{AES-CMAC (RFC 4493):} Los 4 vectores de test oficiales pasan correctamente tras la validación cruzada con la implementación de referencia en Python. Se documentaron 2 errores tipográficos en la propia RFC, confirmados mediante la librería \texttt{cryptography} de Python.

    \item \textbf{ASCON:} Las 3 variantes (Ascon-128, Ascon-128a, Ascon-80pq) superan todas las pruebas de \textit{round-trip}, detección de corrupción y validación con datos de longitud variable (desde vacío hasta 10 KB).

    \item \textbf{Interoperabilidad CMS/PKCS\#7 con OpenSSL:} Se ha verificado la compatibilidad bidireccional: cifrado con la implementación propia y descifrado con OpenSSL, y viceversa. Todas las estructuras ASN.1 y OIDs (tanto estándar como personalizados para ASCON) son correctamente interpretados por ambas partes.
\end{itemize}

\subsection{Tests de Integración}

Los tests de integración, ejecutados contra contenedores Docker reales de PostgreSQL y WFX, validan el funcionamiento de la plataforma en condiciones similares a producción:

\begin{itemize}
    \item \textbf{Persistencia:} Todas las operaciones CRUD sobre \textit{UpdatePacks} y \textit{LaunchTracks} funcionan correctamente, incluyendo consultas con paginación, filtrado y ordenación.
    \item \textbf{Concurrencia:} Las pruebas de 10 inserciones paralelas se completan sin conflictos ni pérdida de datos.
    \item \textbf{Orquestación WFX:} La carga de \textit{workflows}, creación de \textit{jobs} y gestión de estrategias de lanzamiento (directas y por fases) operan correctamente a través de la API de WFX.
\end{itemize}

\subsection{Cobertura de Código}

La ejecución completa de los tests arroja los siguientes porcentajes de cobertura:

\begin{itemize}
    \item \textbf{SymKMS:} 83.0\% de cobertura de código, con más de 150 tests ejecutados.
    \item \textbf{Updates:} 74.4\% de cobertura de código, con más de 70 tests ejecutados.
\end{itemize}

Las \autoref{fig:cover_symkms} y \autoref{fig:cover_updates} muestran los \textit{treemaps} de cobertura de ambos servicios, donde el tamaño de cada bloque es proporcional al volumen de código del paquete y el color indica el nivel de cobertura alcanzado (verde para alta cobertura, rojo para baja).

\begin{figure}[H]
    \centering
    \includegraphics[width=1\textwidth]{imagenes/resultados/test_cover_symkms.png}
    \caption{Treemap de cobertura de código del servicio SymKMS (83.0\%)}
    \label{fig:cover_symkms}
\end{figure}

\begin{figure}[H]
    \centering
    \includegraphics[width=1\textwidth]{imagenes/resultados/test_cover_updates.png}
    \caption{Treemap de cobertura de código del servicio Updates (74.4\%)}
    \label{fig:cover_updates}
\end{figure}

Estos resultados confirman que los componentes desarrollados cumplen con los requisitos de correctitud criptográfica, interoperabilidad con herramientas estándar y fiabilidad en entornos de integración realistas.

\section{Caso de Ejemplo: Actualización con Firma Post-Cuántica y Cifrado ASCON}

Para validar la plataforma en un escenario realista se ha diseñado deliberadamente el siguiente caso: una flota de cuatro dispositivos en la que dos están configurados correctamente, uno carece del certificado de la CA necesario para verificar la firma y otro no dispone de la clave simétrica necesaria para descifrar el paquete. De esta forma se pueden observar tanto los flujos de éxito como los modos de fallo esperados.

\subsection{Preparación del Material Criptográfico}

El primer paso consiste en generar la clave asimétrica que se utilizará para firmar los paquetes de actualización. El desarrollador crea una nueva clave en el KMS de Lamassu (ver \autoref{fig:res_clave_asimetrica}).

\begin{figure}[H]
    \centering
    \includegraphics[width=1\textwidth]{imagenes/resultados/clave_asimetrica.png}
    \caption{Creación de la clave asimétrica para firma digital}
    \label{fig:res_clave_asimetrica}
\end{figure}

A continuación, se genera una solicitud de firma de certificado (CSR) asociada a dicha clave (ver \autoref{fig:res_csr_clave}).

\begin{figure}[H]
    \centering
    \includegraphics[width=1\textwidth]{imagenes/resultados/csr_clave.png}
    \caption{Generación del CSR a partir de la clave asimétrica}
    \label{fig:res_csr_clave}
\end{figure}

El CSR se firma con una Autoridad Certificadora post-cuántica basada en ML-DSA (ver \autoref{fig:res_ca_pq}), demostrando que la plataforma soporta algoritmos post cuánticos.

\begin{figure}[H]
    \centering
    \includegraphics[width=1\textwidth]{imagenes/resultados/ca_post_cuantica.png}
    \caption{CA post-cuántica basada en ML-DSA utilizada para firmar el CSR}
    \label{fig:res_ca_pq}
\end{figure}

El resultado es un certificado digital vinculado a la clave privada (ver \autoref{fig:res_certificado_clave}), que se incluirá en la estructura CMS del paquete de actualización para que los dispositivos puedan verificar su autenticidad.

\begin{figure}[H]
    \centering
    \includegraphics[width=1\textwidth]{imagenes/resultados/certificado_clave.png}
    \caption{Certificado emitido por la CA post-cuántica para la clave de firma}
    \label{fig:res_certificado_clave}
\end{figure}

Seguidamente se crea la clave simétrica ASCON que se empleará para cifrar el contenido del paquete (ver \autoref{fig:res_clave_simetrica}). Esta misma clave debe estar preinstalada en los dispositivos que vayan a recibir la actualización.

\begin{figure}[H]
    \centering
    \includegraphics[width=1\textwidth]{imagenes/resultados/clave_simetrica.png}
    \caption{Creación de la clave simétrica ASCON para cifrado del paquete}
    \label{fig:res_clave_simetrica}
\end{figure}

\subsection{Creación del Paquete de Actualización}

Con el material criptográfico preparado, el desarrollador accede al formulario de creación de actualización donde sube los ficheros de actualización y el descriptor, selecciona la clave de firma junto con el algoritmo y el certificado emitido por la CA, y elige la clave simétrica ASCON para cifrar todos los ficheros (ver \autoref{fig:res_crear_swu}).

\begin{figure}[H]
    \centering
    \includegraphics[width=1\textwidth]{imagenes/resultados/crear_swu_firm_enc.png}
    \caption{Creación del paquete SWU firmado y cifrado}
    \label{fig:res_crear_swu}
\end{figure}

\subsection{Lanzamiento de la Actualización}

Una vez creado el paquete, desde la pantalla de actualizaciones se indica que es necesario crear un lanzamiento para distribuirlo (ver \autoref{fig:res_launch}).

\begin{figure}[H]
    \centering
    \includegraphics[width=1\textwidth]{imagenes/resultados/launch.png}
    \caption{Indicación de creación de lanzamiento desde la pantalla de actualizaciones}
    \label{fig:res_launch}
\end{figure}

El desarrollador configura los parámetros del lanzamiento y lo ejecuta (ver \autoref{fig:res_lanzar_actualizacion}).

\begin{figure}[H]
    \centering
    \includegraphics[width=1\textwidth]{imagenes/resultados/lanzar_actualizacion.png}
    \caption{Configuración y ejecución del lanzamiento de la actualización}
    \label{fig:res_lanzar_actualizacion}
\end{figure}

\subsection{Resultados en los Dispositivos}

Tras el lanzamiento, las actualizaciones llegan automáticamente a los cuatro dispositivos. Los dispositivos 1 y 2, configurados correctamente con el certificado de la CA y la clave simétrica, completan la actualización sin incidencias (ver \autoref{fig:res_dispositivos_correctos}).

\begin{figure}[H]
    \centering
    \includegraphics[width=1\textwidth]{imagenes/resultados/dispositivios_correctos.png}
    \caption{Dispositivos 1 y 2: actualización completada correctamente}
    \label{fig:res_dispositivos_correctos}
\end{figure}

Sin embargo, el dispositivo 3, al que no se le había cargado el certificado de la CA post-cuántica, no es capaz de verificar la firma del paquete y reporta un error de verificación (ver \autoref{fig:res_device3_error}).

\begin{figure}[H]
    \centering
    \includegraphics[width=1\textwidth]{imagenes/resultados/device3_error_verificacion.png}
    \caption{Dispositivo 3: error de verificación de firma por falta del certificado de la CA}
    \label{fig:res_device3_error}
\end{figure}

Por su parte, el dispositivo 4, que no estaba configurado para recibir actualizaciones cifradas, falla al intentar leer el descriptor ya que se encuentra cifrado y no puede procesarlo (ver \autoref{fig:res_error_encriptacion}).

\begin{figure}[H]
    \centering
    \includegraphics[width=1\textwidth]{imagenes/resultados/error_encriptacion.png}
    \caption{Dispositivo 4: error al no poder descifrar los ficheros del paquete}
    \label{fig:res_error_encriptacion}
\end{figure}

\subsection{Visualización del Estado de los Dispositivos}

Al acceder al detalle de cada dispositivo es posible inspeccionar el diagrama de estados del workflow. Los dispositivos 1 y 2 muestran un flujo completo sin errores, habiendo transitado por todos los estados hasta la finalización exitosa (ver \autoref{fig:res_workflow_correcto}).

\begin{figure}[H]
    \centering
    \includegraphics[width=1\textwidth]{imagenes/resultados/workflow_correcto.png}
    \caption{Workflow completado correctamente en los dispositivos 1 y 2}
    \label{fig:res_workflow_correcto}
\end{figure}

En cambio, los dispositivos 3 y 4 presentan un workflow interrumpido en el estado de fallo, mostrando el mensaje de error correspondiente a cada caso (ver \autoref{fig:res_workflow_fallido}).

\begin{figure}[H]
    \centering
    \includegraphics[width=1\textwidth]{imagenes/resultados/workflow_fallido.png}
    \caption{Workflow fallido en los dispositivos 3 y 4 con mensaje de error}
    \label{fig:res_workflow_fallido}
\end{figure}

\subsection{Resumen del Lanzamiento}

Finalmente, desde la pantalla de actualizaciones se puede ver el resumen global del lanzamiento (ver \autoref{fig:res_resultado_final}). La barra de progreso refleja visualmente el resultado: la mitad en azul, correspondiente a los dos dispositivos que completaron exitosamente la actualización, y la mitad en rojo, representando los dos dispositivos que fallaron. Este resumen proporciona al desarrollador una visión inmediata del alcance y los problemas del despliegue.

\begin{figure}[H]
    \centering
    \includegraphics[width=1\textwidth]{imagenes/resultados/resultado_final.png}
    \caption{Resumen del lanzamiento: 2 dispositivos actualizados correctamente (azul) y 2 fallidos (rojo)}
    \label{fig:res_resultado_final}
\end{figure}
