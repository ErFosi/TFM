\section{Desarrollo}

En esta sección se desarrollan los aspectos técnicos y conceptuales que sustentan la plataforma propuesta. En primer lugar se tratarán las bases teóricas de la criptografía simétrica y asimétrica, necesarias para comprender las decisiones de diseño posteriores. A continuación, se explicará la terminología clave y las herramientas empleadas en los flujos de actualización de los dispositivos. Finalmente, se definirá la plataforma de desarrollo propuesta y su modelo de datos.

\subsection{Bases teóricas: criptografía simétrica y asimétrica}

En el mundo de la seguridad digital, la criptografía desempeña un papel fundamental en el cifrado y descifrado de información. Su objetivo principal es asegurar la confidencialidad e integridad de los datos, protegiéndolos contra accesos no autorizados durante la transmisión o almacenamiento. Esta sección ofrece una visión general de los conceptos básicos de la criptografía, centrándose en los dos tipos principales: criptografía simétrica y asimétrica con sus respectivas características, ventajas y desventajas e implementaciones.

En la criptografía, existen varios términos clave que son fundamentales para entender cómo funciona este campo. A continuación, se definen algunos de los conceptos más básicos:

\begin{itemize}
    \item \textbf{Mensaje}: Es la información original que se desea proteger.
    \item \textbf{Criptograma}: Es el resultado del proceso de cifrado, que oculta el mensaje original utilizando algún algoritmo criptográfico.
    \item \textbf{Criptoanálisis}: Es el estudio de los sistemas criptográficos con el fin de encontrar debilidades que permitan recuperar el mensaje original sin conocer la clave de cifrado.
\end{itemize}
Por otro lado, existen principios clave que son esenciales para proteger la información y los sistemas informáticos de accesos no autorizados, alteraciones y destrucción. Estos principios son la base de cualquier estrategia de seguridad y son fundamentales para garantizar la protección de los datos. A continuación, se presentan brevemente estos principios:
\begin{itemize}
    \item \textbf{Confidencialidad:} Este principio asegura que la información es accesible solo para aquellas personas autorizadas a tener acceso. Protege los datos de accesos no autorizados y garantiza la privacidad de la información.
    
    \item \textbf{Integridad:} La integridad se refiere a la exactitud y completitud de la información y los métodos de procesamiento. Este principio asegura que la información no ha sido alterada de manera no autorizada y que se mantiene tal como fue creada, enviada y recibida.
    
    \item \textbf{No repudio:} Garantiza que una vez realizada una transacción, no se pueda negar su autoría, proporcionando evidencia de la participación de las partes involucradas.
    
    \item \textbf{Autenticidad:} La autenticidad garantiza que las transacciones, las comunicaciones y los datos son genuinos y que las identidades de las partes involucradas son verificadas. Este principio evita la suplantación de identidad y asegura que los datos provienen de una fuente legítima.
\end{itemize}